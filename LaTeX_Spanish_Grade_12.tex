

\documentclass{article}

\usepackage[spanish]{babel}
\usepackage[utf8]{inputenc}

\title{Escribir expresiones con variables}
\author{Grado 12}
\date{}

\begin{document}

\maketitle

\section*{Ejercicio 1}

Un trozo de cinta tiene una longitud de $y$ cm. Ashley la cort� en 4 trozos iguales para hacer el cabello de sus 4 hermanas. Escribe la expresi�n para la cantidad de cinta utilizada para cada hermana.

\textbf{Soluci�n:}

Para encontrar la cantidad de cinta utilizada para cada hermana, dividimos la longitud total de la cinta entre el n�mero de hermanas. Entonces la expresi�n es:

\[
\text{Cantidad de cinta utilizada por cada hermana} = \frac{y}{4}
\]

\section*{Ejercicio 2}

Margaret tiene 16 yardas de tela y est� haciendo vestidos para $z$ mu�ecas. Escribe la expresi�n para la longitud de tela utilizada para cada mu�eca.

\textbf{Soluci�n:}

Para encontrar la longitud de tela utilizada para cada mu�eca, dividimos la longitud total de la tela entre el n�mero de mu�ecas. Entonces la expresi�n es:

\[
\text{Longitud de tela utilizada por cada mu�eca} = \frac{16}{z}
\]

\section*{Ejercicio 3}

Olivia ten�a 14 agujas en su estuche de costura, pero perdi� $r$ agujas. Escribe la expresi�n para el n�mero de agujas en su estuche de costura ahora.

\textbf{Soluci�n:}

Para encontrar el n�mero de agujas en el estuche de costura ahora, restamos el n�mero de agujas perdidas al n�mero inicial de agujas. Entonces la expresi�n es:

\[
\text{N�mero de agujas en el estuche de costura ahora} = 14 - r
\]

\end{document}

