
\documentclass{article}
\usepackage[spanish]{babel}
\usepackage[utf8]{inputenc}

\title{Solving Expressions with Variables}
\author{Grade 7}

\begin{document}
\maketitle

\section{Instrucciones}
Este documento contiene una gu�a paso a paso para resolver un conjunto de problemas con variables relacionadas con matem�ticas. A continuaci�n, encontrar�s tres ejercicios seguidos de sus soluciones. Lee cuidadosamente cada problema y utiliza las notas de soluci�n para ayudarte a escribir las expresiones correctas.

\section{Ejercicio 1}
Un pedazo de list�n es de $y$ cm de largo. Ashley lo cort� en 4 partes iguales para hacer peinados a sus 4 hermanas. Escribe la expresi�n para la cantidad de list�n usado para cada hermana.

\textbf{Soluci�n:}
En este problema, la cantidad total de list�n es de $y$ cm, y queremos dividirlo en 4 partes iguales para las 4 hermanas. Para encontrar la cantidad de list�n usado para cada hermana, dividimos la longitud total entre el n�mero de hermanas. Por lo tanto, la expresi�n es:

\[
\frac{y}{4}
\]

\section{Ejercicio 2}
Margaret tiene 16 yardas de tela y est� haciendo vestidos para $z$ mu�ecas. Escribe la expresi�n para la longitud de tela usada para cada mu�eca.

\textbf{Soluci�n:}
En este problema, la cantidad total de tela es de 16 yardas y queremos dividirla entre $z$ mu�ecas. Para encontrar la longitud de tela usada para cada mu�eca, dividimos la cantidad total entre el n�mero de mu�ecas. La expresi�n es:

\[
\frac{16}{z}
\]

\section{Ejercicio 3}
Olivia ten�a 14 agujas en su estuche de costura, pero perdi� $r$ agujas. Escribe la expresi�n para el n�mero de agujas en su estuche ahora.

\textbf{Soluci�n:}
En este problema, la cantidad total de agujas era de 14 y perdi� $r$ agujas. Para encontrar el n�mero actual de agujas en su estuche, restamos las agujas perdidas de la cantidad total original. La expresi�n es:

\[
14 - r
\]

\end{document}
